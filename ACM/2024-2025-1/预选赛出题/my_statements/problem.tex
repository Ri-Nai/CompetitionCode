\documentclass[a4paper, 12pt]{article}
\usepackage{CJKutf8}
\usepackage{ctex}
\usepackage{indentfirst}

\begin{document}
    \title{存在证明的自动机械}
    \author{Reina}
    \date{\today}
    \maketitle
    \section*{Introduction}
    为了抵达命运石之门,Maho 必须研发出一台可行的 \texttt{时光机器}。

    一台 \texttt{机器} 由 \(n\) 个 \texttt{齿轮} 和 \(m\) 条链接这些 \texttt{齿轮} 的
    链条构成

    一台 \texttt{时光机器} 需要满足以下两个条件:

    \begin{enumerate}
        \item \textbf{唯一的二进制地址分配}:每个 \texttt{齿轮} 分配一个唯一的二进制地址,所有 \texttt{齿轮} 的二进制地址长度均为 \(d\),即每个地址由 \(d\) 位二进制数字构成,地址不重复且不存在空地址。

        例如,当 \(d = 3\) 时,可能的地址包括:\texttt{000}, \texttt{001}, \texttt{010}, \texttt{011}, \texttt{100}, \texttt{101}, \texttt{110}, \texttt{111}。为满足要求,我们需要把这些地址分配给 8 个 \texttt{齿轮}

        \item \textbf{链接条件}:两个 \texttt{齿轮} 之间 \textbf{当且仅当} 它们的二进制地址有且只有一位不同,才有链接。

        例如,若 \texttt{齿轮} 的地址为 \texttt{001} 和 \texttt{100},则它们不应有链接;若地址为 \texttt{010} 和 \texttt{000},则应有链接。
    \end{enumerate}

    给定一台 \texttt{机器} 的 \texttt{齿轮} 及其链接关系,你的任务是判断是否能够将长度为某个 $d$ 的所有二进制地址分配给所有 \texttt{齿轮} ,以使该 \texttt{机器} 成为一台 \texttt{时光机器}

    \textbf{\textit{El Psy Kongroo}}

    \section*{Input Format}
    \begin{itemize}
        \item 第一行包含两个整数 \(n\) 和 \(m\) (\(2 \le n \le 2 \cdot 10^5\), \(1 \le m \le 2 \cdot 10^5\)),分别为 \texttt{齿轮} 数量与链接数量。
        \item 以下 \(m\) 行,每行两个整数 \(a\) 和 \(b\) (\(1 \le a, b \le n\), \(a \neq b\)),表示第 \(a\) 和第 \(b\) 个 \texttt{齿轮} 之间有链条链接。
    \end{itemize}

    保证每对 \texttt{齿轮} 之间最多只有一条链条。
    \section*{Output Format}
        输出包含一行
        \begin{itemize}
            \item
            若能构造 \texttt{时光机器} ,则输出一个字符串"\texttt{yes}"
            \item
            否则输出一个字符串"\texttt{no}"
        \end{itemize}

        均不包含引号,区分大小写,"\texttt{Yes}", "\texttt{NO}"等答案均不得分

    \section*{Note}

        对于样例1,我们需要长度为 2 的所有二进制地址。

        我们可以分别为齿轮 1, 2, 3, 4分配 \texttt{00}, \texttt{01}, \texttt{11}, \texttt{10} 使其满足连边的限制

        由于齿轮 3 和 齿轮 4 之间缺少链接,所以本样例答案为"\texttt{no}"


        对于样例2,我们需要长度为 3 的所有二进制地址。
        可以证明以下地址构造为一组可行解

        1: \texttt{000}

        2: \texttt{001}

        3: \texttt{010}

        4: \texttt{110}

        5: \texttt{111}

        6: \texttt{011}

        7: \texttt{100}

        8: \texttt{101}



\end{document}
